% ---
% about:
%   abbreviate_middle: true
%   email: my_email@georgetown.edu
%   family_name: Lastname
%   github_username: my_github_name
%   given_name: Firstname
%   homepage_url: https://me.georgetown.domains
%   institute: Georgetown University
%   middle_name: Middlename
%   phone: 123-456-7890
%   pronouns: My Pronouns
%   scholar_url: https://scholar.google.com/citations?user=my_scholar_id
%   title: My Title
% awards:
% - date: '[Month] [Year]'
%   name: 'Award #1'
% - date: '[Month] [Year]'
%   name: 'Award #2'
% - date: '[Month] [Year]'
%   name: 'Award #3'
% cvconfig:
%   include_job_bullets: true
%   justify_framework_list: false
% education:
%   schools:
%   - degrees:
%     - date: Expected Sep 2024
%       name: MS in Data Science and Analytics
%     loc: Washington, DC, USA
%     name: Georgetown University
%   - degrees:
%     - date: '[Month] [Year]'
%       name: '[Degree Type] in [Major]'
%     - date: '[Month] [Year]'
%       name: '[Second Degree Type] in [Second Major]'
%     loc: '[City], [State], [Country]'
%     name: A Different University
% employment:
% - bullets:
%   - 'Describe your contribution(s) here. You can use \LaTeX\ code here, so long as
%     you make sure to use \textbf{double backslashes} to denote LaTeX commands, like:
%     $N > \textrm{80K}$. You can also use the \texttt{texttt} command to make fixed-width
%     formatted text: \texttt{pandas}, \texttt{statsmodels}, \texttt{bokeh}.'
%   - 'Second bullet, using \texttt{textit}: \textit{Working for Respect: Community
%     and Conflict at Walmart} (Reich and Bearman)'
%   employer: 'Employer #1'
%   end_date: '[Month] [Year]'
%   loc: '[City], [State], [Country]'
%   position: Position 1
%   start_date: '[Month] [Year]'
% - bullets:
%   - First contribution.
%   - Second contribution.
%   employer: 'Employer #2'
%   end_date: '[Month] [Year]'
%   loc: '[City], [State], [Country]'
%   position: Position 2
%   start_date: '[Month] [Year]'
% - bullets:
%   - First contribution.
%   - Second contribution.
%   employer: 'Employer #3'
%   end_date: '[Month] [Year]'
%   loc: '[City], [State], [Country]'
%   position: Position 3
%   start_date: '[Month] [Year]'
% languages:
%   categories:
%   - entries:
%     - Lang 1
%     - Lang 2
%     name: Fluent
%   - entries:
%     - Lang 3
%     - Lang 4
%     - Lang 5
%     name: Working Proficiency
%   - entries:
%     - Lang 6
%     name: Reading Proficiency
% references:
% - email: ref1@example.edu
%   institute: Institute
%   loc: Location
%   name: 'Reference #1'
%   title: Title
% - email: ref2@example.edu
%   institute: Institute
%   loc: Location
%   name: 'Reference #2'
%   title: This Is A Very Long Title
%   title_line2: With Two Lines
% - email: ref3@example.edu
%   institute: Institute
%   loc: Location
%   name: 'Reference #3'
%   title: Title
% - email: ref4@example.edu
%   institute: Institute
%   loc: Lopcation
%   name: 'Reference #4'
%   title: Title
% skills:
%   frameworks:
%   - category: Data Science
%     entries:
%     - Pandas
%     - tidyverse
%   - category: NLP
%     entries:
%     - spaCy
%     - Gensim
%     - tidytext
%   - category: Misc
%     entries:
%     - Git
%     - MongoDB
%     - Qualtrics
%     - Quarto
%   languages:
%   - Python
%   - R
%   - HTML/CSS
% ---
% template.tex
%-------------------------
% Resume in Latex
% Author : Sourabh Bajaj
% Website: https://github.com/sb2nov/resume
% License : MIT
%------------------------
\documentclass[letterpaper,11pt]{article}
\usepackage{amsmath}
\usepackage{amssymb}
\usepackage{mathtools}
\usepackage[T1]{fontenc}
\usepackage{lmodern}
\usepackage{latexsym}
\usepackage{titlesec}
\usepackage{marvosym}
%\usepackage[usenames,dvipsnames]{color}
%\usepackage{fullpage}
%\usepackage[showframe,letterpaper,margin=0.5in]{geometry}
\usepackage[letterpaper,margin=0.5in]{geometry}
%\usepackage{layout}
% Uncomment this to show the frames around each page element, for debugging
%\usepackage{showframe}
\usepackage{verbatim}
\usepackage{enumitem}
\usepackage[pdftex]{hyperref}
\usepackage{changepage}
\usepackage{lastpage}
\usepackage{relsize}
\usepackage{tabularx}
\newcolumntype{b}{X}
\newcolumntype{s}{>{\hsize=.5\hsize}X}
\usepackage{ragged2e}

\usepackage{cprotect}

\usepackage{xcolor}
\hypersetup{
	colorlinks,
	linkcolor={red!50!black},
	citecolor={blue!50!black},
	urlcolor={blue!80!black}
}

\newcommand{\cvitemsep}{2mm}
\newcommand{\emplitemsep}{3mm}

% Adjust margins
%\addtolength{\oddsidemargin}{-0.5in}
%\addtolength{\evensidemargin}{-0.5in}
%\addtolength{\textwidth}{1in}
%\addtolength{\topmargin}{-.5in}
%\addtolength{\textheight}{1.0in}
\setlength{\marginparwidth}{0pt}
\setlength{\marginparsep}{0pt}
\setlength{\footskip}{0pt}
%\setlength{\headsep}{0pt}

\urlstyle{same}

\raggedbottom
\raggedright
%\setlength{\tabcolsep}{0in}

% Sections formatting
\titleformat{\section}{
	\vspace{-4pt}\scshape\raggedright\large
}{}{0em}{}[\color{black}\titlerule \vspace{-5pt}]

%-------------------------
% Custom commands
\newcommand{\resumeItem}[2]{
	\item\small{
		\textbf{#1}{: #2 \vspace{-2pt}}
	}
}

\newcommand{\resumeSubheading}[4]{
	\vspace{-1pt}\item
	%\begin{tabular*}{0.97\textwidth}{l@{\extracolsep{\fill}}r}
	\begin{tabular*}{\textwidth}{l@{\extracolsep{\fill}}r}
		\textbf{#1} & #2 \\
		\textit{\small#3} & \textit{\small #4} \\
	\end{tabular*}\vspace{-5pt}
}

\newcommand{\resumeSubItem}[2]{\resumeItem{#1}{#2}\vspace{-4pt}}

\renewcommand{\labelitemii}{$\circ$}

\newcommand{\resumeSubHeadingListStart}{\begin{itemize}[leftmargin=0pt,label={}]}
\newcommand{\resumeSubHeadingListEnd}{\end{itemize}}
\newcommand{\resumeItemListStart}{\begin{itemize}}
\newcommand{\resumeItemListEnd}{\end{itemize}\vspace{-5pt}}

\usepackage{fancyhdr}
\pagestyle{fancy}
\fancyhf{} % clear all header and footer fields
%\fancyfoot[R]{\thepage}
%\fancyfoot{}
%\rfoot{\textit{Page \thepage\ of \pageref*{LastPage}}}
\fancyfoot[R]{\textit{Page \thepage\ of \pageref*{LastPage}}}
%\fancyfoot{}
\renewcommand{\headrulewidth}{0pt}
\renewcommand{\footrulewidth}{0pt}
%\renewcommand{\footrulewidth}{\textwidth}

\newcommand{\skfill}{\hfill}
\newcommand{\rpkg}[1]{#1}

%-------------------------------------------

\begin{document}
%	\layout
	
	%----------HEADING-----------------
	\begin{tabular*}{\textwidth}{@{}l@{\extracolsep{\fill}}r@{}}
		\textbf{\huge Firstname M. Lastname} & Email: \href{mailto:my\_email@georgetown.edu}{my\_email@georgetown.edu} \\ 
		\url{https://me.georgetown.domains} & \href{tel:123-456-7890}{123-456-7890} $\mathlarger{\cdot}$ Github: \href{https://github.com/my\_github\_name/}{\texttt{my\_github\_name}} \\
	\end{tabular*}


% Education


\section{Education}
\resumeSubHeadingListStart


\vspace{-1pt}\item
%\begin{tabular*}{0.97\textwidth}{l@{\extracolsep{\fill}}r}
\begin{tabular*}{\textwidth}{@{}l@{\extracolsep{\fill}}r@{}}
	\textbf{Georgetown University} & Washington, DC, USA \\
    	\textit{\small MS in Data Science and Analytics} & {\small Expected Sep 2024} \\
    \end{tabular*}\vspace{-5pt}


\vspace{-1pt}\item
%\begin{tabular*}{0.97\textwidth}{l@{\extracolsep{\fill}}r}
\begin{tabular*}{\textwidth}{@{}l@{\extracolsep{\fill}}r@{}}
	\textbf{A Different University} & [City], [State], [Country] \\
    	\textit{\small [Degree Type] in [Major]} & {\small [Month] [Year]} \\
    	\textit{\small [Second Degree Type] in [Second Major]} & {\small [Month] [Year]} \\
    \end{tabular*}\vspace{-5pt}


\resumeSubHeadingListEnd


% Employment


\section{Employment}
\resumeSubHeadingListStart

\vspace{-1pt}
\item 
\begin{tabular*}{\linewidth}{@{}l@{\extracolsep{\fill}}r}
	\textbf{Employer \#1} & [City], [State], [Country] \\
	\textit{\small Position 1} & {\small [Month] [Year] -- [Month] [Year]} \\
		\multicolumn{2}{p{0.99\linewidth}}{
		\parbox{\linewidth}{
		\begin{itemize}[leftmargin=*,topsep=1pt,itemsep=0pt]
			
		{\small \item[\textbullet] Describe your contribution(s) here. You can use \LaTeX\ code here, so long as you make sure to use \textbf{double backslashes} to denote LaTeX commands, like: $N > \textrm{80K}$. You can also use the \texttt{texttt} command to make fixed-width formatted text: \texttt{pandas}, \texttt{statsmodels}, \texttt{bokeh}. }
			
		{\small \item[\textbullet] Second bullet, using \texttt{textit}: \textit{Working for Respect: Community and Conflict at Walmart} (Reich and Bearman) }
				\end{itemize}
		}
	}
	\end{tabular*}\vspace{-5pt}
\item 
\begin{tabular*}{\linewidth}{@{}l@{\extracolsep{\fill}}r}
	\textbf{Employer \#2} & [City], [State], [Country] \\
	\textit{\small Position 2} & {\small [Month] [Year] -- [Month] [Year]} \\
		\multicolumn{2}{p{0.99\linewidth}}{
		\parbox{\linewidth}{
		\begin{itemize}[leftmargin=*,topsep=1pt,itemsep=0pt]
			
		{\small \item[\textbullet] First contribution. }
			
		{\small \item[\textbullet] Second contribution. }
				\end{itemize}
		}
	}
	\end{tabular*}\vspace{-5pt}
\item 
\begin{tabular*}{\linewidth}{@{}l@{\extracolsep{\fill}}r}
	\textbf{Employer \#3} & [City], [State], [Country] \\
	\textit{\small Position 3} & {\small [Month] [Year] -- [Month] [Year]} \\
		\multicolumn{2}{p{0.99\linewidth}}{
		\parbox{\linewidth}{
		\begin{itemize}[leftmargin=*,topsep=1pt,itemsep=0pt]
			
		{\small \item[\textbullet] First contribution. }
			
		{\small \item[\textbullet] Second contribution. }
				\end{itemize}
		}
	}
	\end{tabular*}\vspace{-5pt}
\resumeSubHeadingListEnd


% ------
% Awards
% ------

\section{Awards \& Achievements}

Award \#1 \hfill [Month] [Year]\vspace{1mm} \\
Award \#2 \hfill [Month] [Year]\vspace{1mm} \\
Award \#3 \hfill [Month] [Year]\vspace{1mm} \\

% ======
% Skills
% ======

\section{Technical Skills}

\textbf{Languages}:
Python, R, HTML/CSS \vspace{2mm}\\

\textbf{Frameworks/Packages}:\vspace{1mm}
\setlength\tabcolsep{0.015\textwidth}
\noindent\begin{tabularx}{\textwidth}{@{}bb@{}}
												\textbf{Data Science}: 				Pandas, 				tidyverse				&
				\textbf{NLP}: 				spaCy, 				Gensim, 				tidytext				\vspace{0.5mm}				\\
															\textbf{Misc}: 				Git, 				MongoDB, 				Qualtrics, 				Quarto				&
				\\
		\end{tabularx}

% =========
% Languages
% =========

\section{Languages}

\textbf{[Fluent]} 
Lang 1, Lang 2 
\textbf{[Working Proficiency]} 
Lang 3, Lang 4, Lang 5 
\textbf{[Reading Proficiency]} 
Lang 6 

\clearpage

%--------
% References
%-------
\section{References}

\begin{tabularx}{\textwidth}{@{}bb@{}}

    % We're filling in each 2, so skip if it's an (even) entry
		    						% Name
		\textbf{Reference \#1} &  \textbf{Reference \#2}  \\
		% Title
		\textit{Title} &  \textit{This Is A Very Long Title}  \\
		% (Title line 2, ) Institute
		 Institute &   \textit{With Two Lines}, Institute  \\
		% Location
		Location & Location \\
		% Email
		\href{mailto:ref1@example.edu}{\texttt{ref1@example.edu}} &  \href{mailto:ref2@example.edu}{\texttt{ref2@example.edu}} \\[2ex]
	    % We're filling in each 2, so skip if it's an (even) entry
	    % We're filling in each 2, so skip if it's an (even) entry
		    						% Name
		\textbf{Reference \#3} &  \textbf{Reference \#4}  \\
		% Title
		\textit{Title} &  \textit{Title}  \\
		% (Title line 2, ) Institute
		 Institute &  Institute  \\
		% Location
		Location & Lopcation \\
		% Email
		\href{mailto:ref3@example.edu}{\texttt{ref3@example.edu}} &  \href{mailto:ref4@example.edu}{\texttt{ref4@example.edu}} \\[2ex]
	    % We're filling in each 2, so skip if it's an (even) entry
	
\end{tabularx}

\end{document}
